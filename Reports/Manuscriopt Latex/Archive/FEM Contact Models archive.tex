The shape of a blunt AFM tip, presented in Figure \ref{fig: Capped-Sphere-Plot}A, is a simplified construct similar to the SEM image of actual AFM tips shown by Chen \textit{et al.}\cite{chen_luo_doudevski_erten_kim_2019}. The tip is modelled as a rigid (incompressible) cone with opening angle $\theta = 20^o$ ending in a spherical termination of radius $R$\cite{canet2014correction}. The spherical portion smoothly transitions to the conical segment at the tangential contact point described by Equation \ref{eq: AFM Tip}. Therefore, the tip can be considered spherical for indentations $\delta/R < 1-\sin(\theta) \approx 0.65$, which allows for a direct comparison with analytical indentation models.

The indentation into an elastic sphere provides a test and validation of our FEM approach. Following the common experimental determination of Young modulus\textit{et al.}\cite{sun2021determination,DIMITRIADIS20022798,kontomaris2019determination, kontomaris2020hertz}, theoretical contact models are used to fit the Young modulus for simulated indentations of elastic spheres of varying radii. The elastic sphere moves freely with a fixed, rigid base beneath. Restricting indentation to the z-axis allows the modelling to be asymmetrically centred around the z-axis. 

Indentation of the AFM tip into the soft spherical sample, with elastic modulus set here as $E = 1000$kPa, exerts a compressive force $\mathbf{F}$ onto the sample. Illustrated in Figure \ref{fig: Capped-Sphere-Plot}B, the compression of the sphere enhances the perceived indentation depth, $\delta_{12}$, and indentation force is distributed between the reaction at the base and the indenter such that perceived force, $F_{12}$, is diminished. This necessitates accounting for the indentation between the indenter and the surface and the indentation between the surface and the base. 

The forces and displacement across the AFM tip surface are mapped to a central reference point within ABAQUS. Figure \ref{fig: Capped-Sphere-Plot}B, shows the extracted force-indentation data for various surface-indenter ratios $\frac{r}{R}$. The data is scaled in dimensionless units according to the Hertz equation(\ref{eq: Hertz}), with force ($\frac{F}{R^2E^*}$) and relative indentation ($\frac{\delta}{R}$). The simulations give the characteristic force curve for the elastic indentation. Increasing the surface radius ($R$) leads to a decrease in the bounded area, indicating heightened energy requirements for compressing smaller spheres. This illustrates indentation on smaller surfaces induce significant horizontal shear stress and more extensive compression.

The behaviour of the simulated indentation is characterised and compared with the theoretical indentation models. The Hertz model provides the relation between the applied force, $F$, and the indentation depth, $\delta_{12}$, \cite{hertz1881contact,hertz1882contact,hertz1896contact}

\begin{equation} F_{Hertz}(\delta_{12}) =  \frac{4}{3} \frac{E}{(1-\nu^2)} \sqrt{R^*} \delta_{12}^{3/2} \label{eq: Hertz} \end{equation}

where $E$ is Young’s modulus, $\nu$ is the Poisson’s ratio of the sample, and $R^*$ is the tip-surface contact radius such that $\frac{1}{R^*} = \frac{1}{R} + \frac{1}{r}$, with indenter radius $R$ and surface radius $r$. This relation allows the calculation of Young's modulus as a fitting parameter from measurements of force curves\cite{vinckier1998measuring, kontomaris2018hertz,kontomaris2020hertz}. Applying the Hertzian analysis to conical indenters, we recover the Sneddon model. For spherical samples, a modified Sneddon model developed by Han \textit{et al.}\cite{han2021modified} is given as,

\begin{equation}F_{Sneddon}(\delta_{12}) = \frac{2}{\pi}\frac{E}{(1-\nu^2)} \cdot tan(\theta)\delta^{2} \cdot f\left(\frac{\delta}{2r}\right)\label{eq: Sneddon}\end{equation}

\begin{equation}  f\left(\frac{\delta}{2r}\right) = 1+ \gamma \left(\frac{\delta}{2r}\right) \end{equation}

\begin{equation} \gamma = -5.103\nu^2 - 13.99\nu + 13.53 \end{equation}

where $\theta$ is the indenters principle angle. However, as shown by Glaubitz \textit{et al.}\cite{glaubitz2014novel}, the compression of soft spherical samples during indentation leads to significant underestimations of Young's modulus when analysing the AFM force curves with the simple Hertz model. Therefore, compression is accounted for using a "Double Contact" model for a spherical indenters described by Glaubitz \textit{et al.}\cite{glaubitz2014novel} and by Dokukin \textit{et al.}\cite{dokukin2013quantitative} given by,

\begin{equation}  F_{Double}(\delta_{12}) =  \frac{4}{3} \frac{E}{(1-\nu^2)} \left[ \frac{(R^*r)^{1/3}}{R^{* 1/3}+r^{1/3}} \right]^{3/2}\delta_{12}^{3/2} \label{eq: Hertz Double Contact}\end{equation}

The simple Hertz and Sneddon models describe contact forces for spherically curved elastic half-spaces, which present a linearly elastic response\cite{kontomaris2019determination} and are assumed to extend infinitely in all directions\cite{kontomaris2019harmonic}. Therefore, we expect deviations for these theoretical models as the elastic spheres are finite in extent and present other dynamics responses such as compression. In addition, the Sneddon model describes the behaviour of a conical tip which produces greater curvature and elastic response due to an infinitely fine tip. The Double Contact model more accurately models AFM indentations; however, due to the assumed spherical indenter, we expect the AFM tip to produce a curve with transitional behaviour around $\delta/R=0.65$ as the indenter moves from the spherical portion to the conical. This can produce some deviation. Moreover, these models require the AFM tip radius to be at least ten times smaller than the sample dimension for the assumption to hold. Therefore, we expect large deviations for small surface ratios.

The dimensionless force, $\frac{F}{E^*R^2}$, as it varies with relative indentation depth, $\frac{r}{R}$, is fitted to the theoretical models as shown in Figure \ref{fig: Capped-Sphere-Plot}C. The Double Contact and Hertz model produced qualitatively tight fits. In contrast, the Sneddon model displays more pronounced curvature, as it overestimates deformation and contact forces are distributed more over the spherical AFM tip compared to a sharper conical tip. The accuracy of the models for various surface radii ($r$) can be assessed by quantifying the measured Young's modulus ($E_{AFM}$). Young's modulus is extracted as a fitting parameter from each model and divided by the true Young's modulus of the sample, resulting in a relative elastic modulus ($\frac{E_{AFM}}{E_{Sample}}$) \cite{sun2021determination, DIMITRIADIS20022798, kontomaris2019determination}.

\begin{figure}[htp]
    \centering
    \includegraphics[page=3, trim= 70 465 70 50, clip, width=\linewidth]{Figures/Pictures.pdf} 
    % \includegraphics[width=1\linewidth]{Figures/Capped-Sphere-Manuscript.pdf}
    \caption{\label{fig: Capped-Sphere-Plot} (A) Illustration of double contact experienced by spherical sample. For indentation $\delta$, indenter displacement $\delta_{12}$, surface compression $\delta^*_{32}$, indentation force $F$, indenter reaction force $F_{12}$, and base reaction force $F_{32}$. Corrected values for the indentation depth, $\delta$, are calculated by subtracting the surface compression at its base, $\delta^*_{32}$, from the indenter's displacement $\delta_{12}$. (B) Force curve for indentation $\delta/R$ into the elastic sphere of varying radius, r/R. (C) Plot of fitted contact models over  dimensionless force, $\frac{F}{E^*R^2}$, against relative indentation, $\frac{\delta}{r}$ range. For surface-indenter ratios, $r/R = 3.0$. (D) Variation of relative Young's Modulus, $\frac{E_{AFM}}{E_{Sample}}$ for varying surface-indenter ratios, $r/R$.}
\end{figure}

The results in Figure \ref{fig: Capped-Sphere-Plot}D indicate that the simple Hertzian model underestimates the Young's modulus as the data converged below the sample value. As the indentation necessitates accounting for the indentation between the indenter and the surface and the compression of the surface into the base, Hertz-fitted models underfit by approximately half. This reduction in fitted Young's modulus is consistent with the reduction in the reaction force experienced by the indenter when compressing the sample. This highlights the importance of the contribution of compression at the base of the sample. The Sneddon model shows poor fit as it converges to 0. This indicates that the spherical indentation is dominant at these scales, which is consistent with the relative size of the spherical termination. 

In comparison, the Double Contact models converge to the sample Young's modulus within the confidence interval. At small surface radii, excessive amounts of compression cause significant errors and deviation from the theoretical models. As the AFM tip is comparable to the surface, the force is distributed over a more significant portion of the surface and causes buckling and shear forces. In addition, this can create larger reaction forces at the base. Consequently, this creates deviation from the theoretical models along with the contradiction of the assumption of an elastic half-space with an infinite extent. The error decreases as the radius of the surface increases and compressive effects lessen. A minor offset is expected because the non-spherical portion of the indenter produces less curvature and deviation from the theoretical model. Fitting data with a restricted indentation depth, excluding the conical section at depths greater than $\delta/R=0.65$, the force-curves highlight the transitional behaviour. As shown by the dashed line in Figure \ref{fig: Capped-Sphere-Plot} when considering indentations with only the spherical portion the Double Contact model produces tighter fits at small indenter-surface ratios. Overall, these results further validated our ABAQUS modelling.