\subsection{Simulation of AFM Imaging}

\subsubsection{Pseudo Scan Dynamics for Imaging}

ABAQUS simulations replicate AFM raster-scans by performing independent indentations across the surface. Scan positions are determined by subdividing the XY domain, and calculate corresponding initial indenter heights. Extracting simulated vertical forces and displacements produces a four-dimensional array of indenter positions and forces. Subsequently, contours are computed using a reference force, generating the final AFM images. Biological structures are produced using Protein Data Bank (PDB) files that specify the biomolecules constituent atoms and corresponding coordinates. As the simplest approach, the biomolecule are modelled as an elastic material produced from the assembly of spheres (with van der Waals radius) of the individual atoms. The structure is assumed to be a continuous, homogeneous and isotropy elastic material. The molecule is partially embedded in a rigid base/ substrate and fixed at its base to simulate a soft molecule absorbed onto a solid support. 

Initial heights are computed from hard-sphere/ tangential contact points, ensuring consistent indentation depths across the surface. The tangent points between the surfaces are calculated by setting the tip above the sample and determining the minimum vertical distance between the tip and the molecule's surface. Figure \ref{fig: ABAQUS Model-Setup}D illustrates the calculations made. By computing the vertical distances between the indenters surface and atoms within the indenter's boundary ($R_{Boundary}$), an array of height differences ($\Delta Z$) is obtained. As illustrated in Figure \ref{fig: ABAQUS Model-Setup}C, the minimum $\Delta Z$ value corresponds to the tangential contact position. Repeating this process for all scan locations, resulting in an array of coordinates. For computational efficiency, only positions where the tip and molecule interact are included.



\subsubsection{Image Processing and Interpolation}

Pseudo-AFM images are generated from two-dimensional arrays of surface heights across the XY grid of scan positions. The heights map the surface contour of equal indentation force over the scan. Contours are calculated from force-indentation data via list comprehension, extracting depth at which the indentation force exceeds a given reference force. Subsequently, visualisations are produced using a normalised Colormap to illustrate the variation in surface heights. Linear or power normalisation is applied depending on detail contrast. Moreover, to increase pixel density images are interpolated using bi-cubic interpolation.

