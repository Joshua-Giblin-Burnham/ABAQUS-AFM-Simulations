Atomic Force Microscopy (AFM) is a versatile three-dimensional topographic technique implementing a mechanical probe to raster-scan and image sample surfaces. The technique provides reliable nanometer measurements of materials and has become a valuable tool with a diverse range of applications. However, there are limited computational recreations of AFM imaging, and the area could benefit from greater tools to aid in interpreting surface characteristics. Consequently, this research presents novel computational modelling of AFM imaging using Finite Element Modelling (FEM). Validation of the modelling focused on the indentation of elastic spheres. The results indicate that simple Hertzian models underestimate the elastic modulus of spherical samples and require Double Contact models. FEM was applied to analyse the perceived compression of simple hemispheres and periodic surfaces during AFM imaging. These simulations highlighted the dependency of the elastic behaviour on the contact radius and tip convolution. Our results indicated that larger indenters require larger forces to compress the sample to the same extent. In addition, Fourier analysis of the simulated AFM contours elucidated a possible novel trend, that larger indentation forces recover more of a surface's periodicity. These simulations show the viability of the FEM approach in reproducing the AFM dynamics and provide a wealth of extensions to be explored.