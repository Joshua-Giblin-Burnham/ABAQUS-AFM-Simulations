% Language setting
\usepackage[english]{babel}

% Set page size and margins
\usepackage[a4paper,top=2cm,bottom=2cm,left=2.5cm,right=2.5cm,marginparwidth=1.5cm]{geometry}

%%%%%%%%%%%%%%%%%%%%%%%% Useful packages %%%%%%%%%%%%%%%%%%%%%%%%%%%%%%%
 
\usepackage[page,toc,titletoc,title]{appendix}
\usepackage{amsmath}
\usepackage{graphicx}
\usepackage{SIunits}
\usepackage{float}
\usepackage[export]{adjustbox}
\usepackage[colorlinks=true, allcolors=black]{hyperref}
\usepackage[nottoc,numbib]{tocbibind}
\usepackage{nameref}
\usepackage{pdfpages}
\usepackage{parskip}
\usepackage{float}
\usepackage{url}
\usepackage{array}
\usepackage{multirow}
\usepackage{csquotes}
\usepackage{fancyhdr}
\usepackage{multicol}
\usepackage{verbatim}
\usepackage{tikz}
\usepackage{setspace}

%%%%%%%%%%%%%%%%%%%%%%%%%% Caption Setup %%%%%%%%%%%%%%%%%%%%%%%%%%%%%%
% \usepackage[format=plain, font=it]{caption}
\usepackage{caption} 
\usepackage{subcaption}

\renewcommand{\thesubfigure}{\textbf{\Alph{subfigure}} }
\DeclareCaptionLabelFormat{plain}{#2}

\captionsetup[figure]{labelfont=bf, font=small}
\captionsetup[subfigure]{labelformat=plain, justification=raggedright, singlelinecheck=false}


%%%%%%%%%%%%%%%%%%%%%%%% Biblography setup %%%%%%%%%%%%%%%%%%%%%%%%%%%%%
\usepackage[backend = biber, 
            style = nature,
            sorting = none ]{biblatex}
\addbibresource{references.bib}
\renewcommand{\cite}[1]{\supercite{#1}}
\urlstyle{same}


%%%%%%%%%%%%%%%%%%%%%%%%%%% Misc Setup %%%%%%%%%%%%%%%%%%%%%%%%%%%%%%%%%
\setlength{\arrayrulewidth}{0.3mm}
\setlength{\tabcolsep}{13pt}
\renewcommand{\arraystretch}{2.5}

\definecolor{codegray}{gray}{0.9}
\newcommand{\code}[1]{\colorbox{codegray}{\texttt{#1} } }

\DeclareUnicodeCharacter{2212}{-}

%%%%%%%%%%%%%%%%%%%%%%%% Word Count Setup %%%%%%%%%%%%%%%%%%%%%%%%%%%%%%
%TC:group table 0 1
%TC:group tabular 1 1

\newcommand{\detailtexcount}[1]{%
  \immediate\write18{texcount -merge -sum -q #1.tex output.bbl > #1.wcdetail }%
  \verbatiminput{#1.wcdetail}%
}
\newcommand{\quickwordcount}[1]{%
  \immediate\write18{texcount -1 -sum -merge -q #1.tex output.bbl > #1-words.sum }%
  \input{#1-words.sum} words%
}

