\subsection{Conclusion}

In conclusion, our FEM approach has demonstrated some novel and varied applications for the analysis of AFM imaging. Our analysis of the contact models for elastic half-spaces and spheres agreed with the theoretical models. Most prominently, the results highlight the under-fitting of the elastic modulus produced by simple Hertzian models for spherical samples. Moreover, our novel formulation of the Double Contact model for conical indenters demonstrated good predictive power over a range of surface radii.  

 Moreover, applying FEM to analyse the compression of hemispheres and simple periodic surfaces highlighted the quantitative power of this approach. These simulations highlighted the dependency of the elastic behaviour on the contact radius and tip convolution. Our results indicated that larger indenters require larger forces to compress the sample to the same extent. In addition, Fourier analysis of the simulated AFM contours elucidated a possible novel trend that larger indentation forces recover more of a surface's periodicity. 

Finally, applying FEM to simulate the AFM appearance of B-DNA Dodecamer provides promising results. The code provides a range of parameters to allow simulations to emulate real AFM system specifications. These initial simulations showed the viability of this modelling and provided various extensions to be explored. However, this has shown some limitations to this approach, including the simulation times and scalability. For example, these simulations took around 48-72 hours for around 40 scan positions and one week for the 108 scan positions. These time scales limit the effectiveness of this approach for larger biomolecules. In addition, due to the numerical methods used by ABAQUS, deformation is limited and complex geometry simulations can often fail. This affects the repeatability and reliability of the simulation. 

