\subsection{ABAQUS Finite Element Modelling}

FEM subdivides the geometry into small, discrete finite elements, producing a surface mesh. The dynamics are approximated over these finite elements and result in a system of algebraic equations. The system is then modelled using the assembled equations for the finite elements and solutions are approximated via the calculus of variations and minimising an associated error function. Producing AFM images requires the calculation of contours of constant indentation force. Using FEM, the sample surface and probe tip geometry are recreated, and individual indentations across the sample are simulated. This provides 4-dimensional arrays of indenter position and corresponding indentation force. From this, contours of constant force can be used to return the surface contour. Our implementation of FEM utilises the commercial engineering software ABAQUS. To provide accessible simulations of AFM imaging, our implementation of ABAQUS utilised Python scripts to produce simulations. Geometric dimensions are defined in $\text{\AA}$ and forces in $\text{pN}$.

\subsubsection{Modelling AFM Tip Structure}

The AFM probe tip is modelled as a rigid capped conical indenter where the indentation is non-slip and without adhesion. The shape of a blunt AFM tip, presented in Figure \ref{fig: ABAQUS Model-Setup}A, is a simplified construct with a rigid cone with opening angle $\theta$ ending in a spherical termination of radius $R$. The spherical portion smoothly transitions to the conical segment at the tangential contact point described by

\begin{equation}\label{eq: AFM Tip}
\begin{split}
    X_{tangent} = R\cos\theta 
    \\
    Y_{tangent} = R(1-\sin\theta) 
\end{split}
\end{equation}

\subsubsection{Model Formulations and Validation}

All simulations are conducted using the Abaqus code (2017). Computations are quasi-static, in which an implicit algorithm is utilised. The samples are free to move under the interplay of exterior contact forces and there own inertia (translational and rotational). The elastic constitutive relations are integrated with the main body of the Abaqus code through the interface of user material subroutines, Umat in Implicit algorithm. To eliminate the hourglass effect, complete integration element R3D10 tetrahedral elements are employed. Surfaces are assumed to be homogeneous and isotropic with a relative Young's modulus and Poisson ratio comparable to biomolecules. Indentations are simulated with a rigid, indenter and "surface to surface" type contact. The contact was set as "hard", nonadhesive contact in the normal direction and "rough" Coulomb friction (non-slip) in the tangential direction. Boundary conditions fix the base of the structures, and vertical force and indentation data are mapped and sampled via reference points at the centre of the indenter.

Applying ABAQUS to assessing simple contact models of indentation provides a robust validation of the FEM approach. Simulations focused on the indentation of elastic spheres of varying radii. Through comparison with theoretical contact models simulations provide verification of the simulations accuracy. The behaviour of indentation is characterised and compared with the theoretical indentation models: Hertz\cite{kontomaris2018hertz} for spherical indenters, and Sneddon\cite{han2021modified} for conical indenters. For elastic spheres, Double Contact Models\cite{dokukin2013quantitative,glaubitz2014novel} are required to account for more complex dynamics discussed in the results. For computational efficiency, asymmetric models centered around the indentation axis (y-axis) are implemented. The simulation data was exported to Python and scipy.curve\_fit module was used to fit the desired contact model.


\begin{figure}[H]
    \centering
    % \includegraphics[width=1\linewidth]{Figures/Manuscript Methodology.pdf}  
    \includegraphics[page=2, trim= 70 350 70 50, clip, width=\linewidth]{Figures/Pictures.pdf}   
    \caption{\label{fig: ABAQUS Model-Setup} (A) Illustration of AFM tip geometry as a rigid cone with opening angle $\theta$ ending in a spherical termination of radius $R$. (B) ABAQUS model assembly from elastic indentation tests. Modelled asymmetrically using elastic spheres modelled as semi-circles with a rigid base beneath. (C-D)Schematic diagrams illustrating the calculation of initial scan heights in AFM code. (C) Illustrates the calculation atoms on the surface within the indenters boundary. Black dots represent the XY grid of scan positions. The calculation is restricted to the XY plane in which only atoms inside the radial extent of the indenters, $R_{Boundary}$ (blue), are calculated. The red atom represents an atom outside the extent and thus is omitted, as appose to the green atom, which is included. (D) Illustrates calculation of heights for each given position. An array of all distances ($\Delta Z$) between the indenters and molecules surfaces are calculated (red). The minima of these distances thus give the translation distance to place the indenter in tangential contact.} 
    
\end{figure}
