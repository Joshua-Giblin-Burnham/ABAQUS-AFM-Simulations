



% \begin{itemize}
%     \item Work by Amyot R, Flechsig H  \textit{et al.}. \cite{amyot2020bioafmviewer} produced the BiomolecularAFMviewer in 2020.
%     \item In follow-up work in 2022, Amyot R, Flechsig H  \textit{et al.}.\cite{amyot2022simulation} used the software to reconstruct resolution-limited experimental images as an example of the application of the software.
%     \item Glaubitz \textit{et al.}\cite{glaubitz2014novel} The use of the simple Hertz model for the analysis of Atomic Force Microscopy (AFM) force–distance curves measured on soft spherical cell-like particles leads to significant underestimations of the objects Young's modulus E. To correct this error, a mixed double contact model was derived.
%     \item Double contact also acounted for by Dokukin \textit{et al.}\cite{dokukin2013quantitative}
%     \item A few Finite element modelling papers\cite{zheng2012finite,senda2016computational}
%     \item Simulations for conical indenter from Kontomaris \textit{et al.}\cite{kontomaris2020hertz}
%     \item Ding \textit{et al.}\cite{ding2018elastic} Contrary to the majority of reported results, we find that the elastic modulus of a cell could be independent of indent depths if surface tension is taken into account. Our model seems to be in agreement with experimental data available. A comprehensive comparison will be done in the future.
%     \item Kontomaris \textit{et al.}\cite{kontomaris2018discussion} A paper that investigate the errors that arise under the assumption that a cylindrical/spherical shaped sample is a half space and to provide a simplified methodology to reduce these errors. The new approach can be applied to AFM nanoindentation experiments on cylindrical or spherical biological sample
%     \item Han \textit{et al.}\cite{han2021modified} Experimental tests with different conical indenters have demonstrated that the new model is capable to reliably determine the Young’s modulus of the spherical samples.
%     \item To study of indention in AFM; Liu \textit{et al.}\cite{liu2019finite} validated a FEM model for AFM indention with less than 10\% error when comparing the simulated force-indentation curves with the experimental data.
%     \item Similar analysis of experimental data with FEM done by Roduit \textit{et al.}\cite{roduit2009stiffness}
%     \item The work by Rajabifar \textit{et al.}.\cite{rajabifar2021fast} simulated the viscoelasticity contact between an AFM tip and a surface.
% \end{itemize}